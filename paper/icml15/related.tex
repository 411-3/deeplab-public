\section{Related Work}
Deep Convolutional Neural Networks (DCNNs) have demonstrated state-of-the-art performance on semantic image segmentation tasks \citep{chen2014semantic, eigen2014predicting, farabet2013learning, long2014fully, mostajabi2014feedforward}. Typically, the last few fully connected layers in the network are replaced with convolutional layers followed by upsampling \citep{long2014fully, papandreou2014untangling} to give dense pixel classification. The network parameters are tuned to optimize the pixel-wise segmentation accuracy and typically full, pixel-wise annotations are required to learn the models. However, it is very labor-intensive to collect the full annotations for thousands or millions of images. On the other hand, it is relatively easy to collect weak annotations, such as image-level labels (i.e., information about which object classes are present) or bounding boxes (i.e., coarse object locations).

Training classifiers with only image-level labels for semantic image segmentation has been a challenging problem in the literature. \citet{duygulu2002object} learned the mapping between image regions and image labels by an EM-type algorithm. \citet{verbeek2007region} combined Probabilistic Latent Semantic Analysis and Markov Random Field to assign labels to square patches. A generalized framework to exploit label correlations between different images has been introduced in a series of papers \citep{vezhnevets2010towards, vezhnevets2011weakly, vezhnevets2012weakly}. \citet{xu2014tell} formulated the problem of segmentation using image-level labels as structured prediction with latent variables. Our model is most related to \citet{pathak2014fully, pinheiro2014weakly}, which also employ fully convolutional networks for weakly supervised semantic segmentation. They both employed Multiple Instance Learning (MIL) to leverage the fact that there must be at least one pixel that assumes the weak annotation. \citet{pathak2014fully} estimated the particular pixel from the DCNN output by taking the maximum operation over image positions, while \citet{pinheiro2014weakly} adopted Log-Sum-Exp (LSE) function to approximate the max function. In contrast, our model employs a modified EM algorithm, featuring a censored E-step which takes into account the weak labels when inferring the latent image segmentations. Moreover, \citet{pinheiro2014weakly} proposed to smooth the prediction results by region proposal algorithms (e.g., MCG \citep{arbelaez2014multiscale}), which have been trained with fully annotated images. Instead of exploiting fully annotated images via some region proposal algorithms, we show that directly training our model with a combination of few strongly labeled and many weakly labeled images can achieve state-of-the-art results.

Another form of weak supervision, bounding box annotations, has been utilized for semantic segmentation as well. \citet{xia2013semantic} generated segment hypotheses by CPMC \citep{carreira2012cpmc} within bounding boxes from object detectors, and graph-cut were employed to refine the final segment. \citet{guillaumin2014imagenet} proposed a recursive scheme to propagate segmentation by exploiting both image-level labels and bounding box annotations in ImageNet \citep{deng2009imagenet}. \citet{chen2014beat} attained human-level accuracy for car segmentation by taking use of 3D bounding boxes. \citet{zhu2014learning} proposed latent expectation loss Support Vector Machine to score the positiveness of region proposals within bounding boxes. On the other hand, bounding box annotations are common in the community of interactive segmentation \citep{lempitsky2009image, rother2004grabcut, WuMilcut}. In this work, we show that given bounding box annotations, our model is able to reach the accuracy with model trained by fully annotated images.

From the aspect of weakly supervised learning, our work also bears similarity to several works. To name a few, \citet{oquab2014weakly} fine-tuned an ImageNet-pretrained DCNN on PASCAL VOC image classification task. Their method is able to localize objects without any strong supervision. \citet{Hoffman14Lsda} proposed to learn the adaptation from image classification to object detection for classes without bounding box annotations. \citet{cour2011learning} proposed a convex learning algorithm by extending multi-class loss functions to handle problems with partial labels, in which instance label is not available (but only a candidate set of labels). \citet{Lu2013} proposed an EM algorithm to train a model with weak labels for tracking and identifying sports players in videos.
